
\section{Representation}
\label{sec:rep}

We use \emph{procedural extrusions}~\shortcite{kelly2011interactive}, to create our mass-models:

\begin{itemize}
    \item Procedural extrusions are a parametrization of architecture into a 2D plan, and a set of 2D profiles associated with the edges of this plan. They allow arbitrary angled walls and roofs. We show how to fit them to the data sources.
    \item Procedural extrusions are unstable when adjacent edges in the plan are near-parallel and near-adjacent.
    \item Procedural extrusions with arbitrary plans and profiles are not guaranteed to terminate; we introduce measures to ensure they do.
    \item Procedural extrusions do not always create rectangular faces, as many shape grammars do.
    \item Upon each PE output face we use a shape-grammar~\cite{muller2006procedural} \{within a rectangular region\}to create additional detail.
    \item PEs have fewer parameters than shape-grammars, and are better suited to reconstruction and optimisation.
    \item PEs are able to express more complex architectural forms than Manhattan or cube-world reconstruction techniques.
\end{itemize}

Image: comparison of PEs to i) Manhattan techniques, ii) extruding GIS footprints iii) Shapegrammars?!
