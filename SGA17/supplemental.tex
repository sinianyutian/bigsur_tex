\documentclass{article}


\usepackage{graphicx}
 \usepackage[margin=2cm]{geometry}
\date{}

 \title{Supplemental Material}
 
  
\begin{document}
\graphicspath{{../images/megafacades/}}
\maketitle{}

\begin{figure}[h]
    \centering
    \includegraphics[width=\linewidth]{../images/results/results_pink.png}
    \caption{The input meshes for Figure~15 in Section 6 of the main paper.}
    \label{fig:results}
\end{figure}

\begin{figure}
   \centering
  \def\svgwidth{0.5\linewidth}  
   \input{../images/megafacades/reference.pdf_tex}
  \caption{Reference images for Figure~15. Left column: Satellite aerial images. Right column: Orthographic photographic images. Rows: Oviedo, Manhattan and London. The Manhattan orthographic image is most likely photographic textures rendered onto a mesh.}
  \label{fig:reference}
\end{figure}

\begin{figure}
    \centering
    \includegraphics[width=\linewidth]{../images/results/oxford_sources.png}
    \caption{A visualization of some of the input data for Figure~16. The projected image planes are shown in green, the image centers as orange cubes, and the coarse meshes in pink. Note the intersections between the meshes and the image planes.}
    \label{fig:results}
\end{figure}


\end{document}
