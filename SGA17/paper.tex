\documentclass[acmtog]{acmart}

\acmSubmissionID{0107}

% TOG prefers author-name bib system with square brackets
\citestyle{acmauthoryear}
\setcitestyle{square}

%\documentclass[review]{acmsiggraph}

\usepackage{graphicx}
\usepackage{amsmath}
\usepackage{import}
\usepackage{fixmath}
\usepackage[inline]{enumitem}
\usepackage{acronym}
\usepackage{xspace}
\usepackage{wrapfig}
\usepackage{import}
\usepackage{color}
\usepackage{microtype}
\usepackage{caption}
\usepackage[normalem]{ulem}

%% This causes each citation to link back to the place where it was cited. 
% \usepackage[hyperref]{backref}
%\PassOptionsToPackage{hyphens}{url}\usepackage{hyperref}

%%% Title of your article or abstract.

%\title{Wide Area Urban Data Fusion}
%\title{Inverse Urban Modeling using Large-scale Data Fusion}
%\title{Structured Urban Reconstruction}
\title{BigSUR: Large-scale Structured Urban Reconstruction}

\keywords{urban modeling, structure, reconstruction, \facade parsing and element classification, procedural modeling}

\author{Tom Kelly}
\affiliation{%
 \institution{University College London}
 \country{UK}
 }
 \author{John Femiani}
 \affiliation{
 \institution{Miami University}
 \country{USA}
 }
 \author{Peter Wonka}
 \affiliation{
 \institution{KAUST}
 \country{KSA}
 }
 \author{Niloy J. Mitra}
 \affiliation{
 \institution{University College London}
 \country{UK}
 }

%%% Used by the ``review'' variation; the online ID will be printed on 
%%% every page of the content.


% User-generated keywords.



%%% The next five lines define the rights management block on the first page.
%%% Replace them with the LaTeX commands provided when the form has been completed.

\definecolor{darkgreen}{RGB}{0, 64, 0}

\newcommand{\neu}[1]{{#1}}
\renewcommand{\sout}[1]{}

\newcommand{\te}[1]{\text{\emph{#1}}}
\newcommand{\facade}{fa\c{c}ade\xspace}
\newcommand{\Facade}{Fa\c{c}ade\xspace}
\newcommand{\facades}{fa\c{c}ades\xspace}
\newcommand{\mypara}[1]{\noindent {\bf #1.}}

%variables for optimisation
\newcommand{\isEdge}{s}    %{\te{d}}
\newcommand{\isDifferent}[1]{\ensuremath{\text{\textsc{isDifferent}}(#1)}}
\newcommand{\hasProfileEdge}[1]{\ensuremath{\te{isSweepEdge}(#1)}}
\newcommand{\clusterArea}{\textsc{meshClusterThreshold}}    %{\te{d}}

\newcommand{\edgeNotProfile}{\te{g}}
\newcommand{\profileNotEdge}{\te{u}}
\newcommand{\edgeNoFacade}{\te{l}}
\newcommand{\isFacade}{\te{q}}
\newcommand{\differentProfiles}{\te{r}}
\newcommand{\height}{\te{h}}
\newcommand{\outputM}{structured model}
\newcommand{\outputMs}{structured models\xspace}
\newcommand{\edge}{\ensuremath{\mathbf{e}}}
\newcommand{\htLeft}{\te{htLeft}}
\newcommand{\htRight}{\te{htRight}}
\newcommand{\distance}{\te{distance}}
\newcommand{\cluster}{\mathbb{C}}


% terminology

\acrodef{GSV}{Google Street\-View}
\acrodef{CNN}{Convolutional Neural Network}
\acrodef{GIS}{Geospatial Information System}
\acrodef{IoU}{Intersection over Union}
\acrodef{VR}{Virtual Reality}

\newcommand{\GSV}{\ac{GSV}\xspace}
\newcommand{\GSVs}{\acp{GSV}\xspace}
\newcommand{\GIS}{GIS\xspace}

\newcommand{\streetI}{street-level imagery\xspace}
\newcommand{\StreetI}{Street-level imagery\xspace}
\newcommand{\streetimage}{street-level image\xspace}
\newcommand{\streetimages}{street-level images\xspace}

\newcommand{\GISd}{GIS footprint\xspace}
\newcommand{\GISds}{GIS footprints\xspace}

\newcommand\figref{Figure~\ref}

\newcommand{\vnudge}{\vspace*{-.05in}}

\newcommand{\sweepedge}{sweep-edge\xspace}
\newcommand{\sweepedges}{sweep-edges\xspace}
\newcommand{\Sweepedges}{Sweep-edges\xspace}

\newcommand{\cleanprofiles}{clean-profiles\xspace}
\newcommand{\cleanprofile}{clean-profile\xspace}

\newcommand{\rawprofiles}{raw-profiles\xspace}
\newcommand{\rawprofile}{raw-profile\xspace}

\newcommand{\buildingfacades}{building-fa\c{c}ades\xspace}
\newcommand{\buildingfacade}{building-fa\c{c}ade\xspace}

\newcommand{\horizontalline}{horizontal-line\xspace}
\newcommand{\horizontallines}{horizontal-lines\xspace}

\newcommand{\block}{building-block\xspace}
\newcommand{\blocks}{building-blocks\xspace}

\newcommand{\prominentface}{prominent-face\xspace}
\newcommand{\prominentfaces}{prominent-faces\xspace}

\newcommand{\seedline}{seed-line\xspace}
\newcommand{\seedlines}{seed-lines\xspace}

\newcommand{\buildingfacadepoints}{building-fa\c{c}ade-points\xspace}
\newcommand{\buildingfacadepoint}{building-fa\c{c}ade-point\xspace}

\newcommand{\footprintpolygon}{footprint-polygon\xspace} % or plan polygon? to match plan-edge
\newcommand{\footprintpolygons}{footprint-polygons\xspace}
\newcommand{\Footprintpolygons}{Footprint-Polygons\xspace}

\newcommand{\planedge}{plan-edge\xspace}
\newcommand{\planedges}{plan-edges\xspace}

\newcommand{\groundplane}{ground plane\xspace}

 
\newcommand{\softedge}{soft-edge\xspace}
\newcommand{\softedges}{soft-edges\xspace}

\newcommand{\face}{face\xspace} % or regions? polygons?
\newcommand{\faces}{faces\xspace}

\newcommand{\lidar}{LiDAR\xspace}

\newcommand{\LondonRS}{Little Portland Street}
\newcommand{\LondonOC}{Oxford Circus}

\renewcommand{\sout}[1]{}

% used vars
%ax
%b
%cx
%dx
%ex
%fx
%gx G
%hx
%ix
%jx
%kx
%l
%mx
%nx
%ox
%px
%q
%rx
%sx
%tx
%u
%vx V
%wx
%xx
%yx
%zx


%%% Start of the document.

\begin{document}

\setcopyright{acmlicensed}
\acmJournal{TOG}
\acmYear{2017}\acmVolume{36}\acmNumber{6}\acmArticle{204}\acmMonth{11} \acmDOI{10.1145/3130800.3130823}

\graphicspath{{../images/megafacades/}}
%%% This is the ``teaser'' command, which puts an figure, centered, below 
%%% the title and author information, and above the body of the content.

 
 \begin{teaserfigure}
 \centering
   \hspace*{-.02\textwidth}  
   \def\svgwidth{1.02\textwidth}  
    \input{../images/megafacades/teaser.pdf_tex}
   \caption{{\it Structured Urban Reconstruction.} 
    Given \streetI, \GISds, and a coarse 3D mesh (left), we formulate a global optimization to automatically fuse these noisy, incomplete, and conflicting data sources to create 
   building footprints (middle: colored horizontal polygons) with profiles (vertical ribbons shown for several footprints) and attached building \facades  ~(vertical rectangles). The output encodes a structured  urban model~(right) including the walls, roof, and associated building elements (e.g., windows, balconies, roof, wall color, etc.). 
   Inset below: A reference aerial image. 
   }
   \label{fig:teaser}
\end{teaserfigure}


\begin{abstract}
The creation of high-quality semantically parsed 3D models for dense metropolitan areas is a fundamental urban modeling problem.
%
Although recent advances in acquisition techniques and processing algorithms have resulted in large-scale imagery or 3D polygonal reconstructions, such data-sources are typically noisy, and incomplete, with no semantic structure.  
% 
%
In this paper, we present an automatic data fusion technique that produces high-quality {\outputM}s of city blocks.
%
From coarse polygonal meshes, \streetI, and \GISds, we formulate a binary integer program that globally balances sources of error to produce semantically parsed mass models with associated \facade elements. 
We demonstrate our system on four city regions of varying complexity; our examples typically contain densely built urban blocks spanning hundreds of buildings. In our largest example, we produce a \outputM\ of 37 city blocks spanning a total of 1,011 buildings at a scale and quality previously impossible to achieve automatically.  
\end{abstract}





%
% The code below should be generated by the tool at
% http://dl.acm.org/ccs.cfm
% Please copy and paste the code instead of the example below. 
%
\begin{CCSXML}
<ccs2012>
<concept>
<concept_id>10010147.10010178.10010224.10010225.10010227</concept_id>
<concept_desc>Computing methodologies~Scene understanding</concept_desc>
<concept_significance>500</concept_significance>
</concept>
<concept>
<concept_id>10010147.10010371.10010396.10010402</concept_id>
<concept_desc>Computing methodologies~Shape analysis</concept_desc>
<concept_significance>500</concept_significance>
</concept>
<concept>
<concept_id>10010147.10010371.10010396.10010397</concept_id>
<concept_desc>Computing methodologies~Mesh models</concept_desc>
<concept_significance>300</concept_significance>
</concept>
<concept>
<concept_id>10010405.10010469.10010472</concept_id>
<concept_desc>Applied computing~Architecture (buildings)</concept_desc>
<concept_significance>300</concept_significance>
</concept>
</ccs2012>

\end{CCSXML}

\ccsdesc[500]{Computing methodologies~Scene understanding}
\ccsdesc[300]{Computing methodologies~Shape analysis}
\ccsdesc[300]{Computing methodologies~Mesh models}
\ccsdesc[500]{Applied computing~Architecture (buildings)}


%
% End generated code
%

% The next three commands are required, and insert the user-generated keywords, 
% The CCS concepts list, and the rights management text.
% Please make sure there is a blank line between each of these three commands.

\maketitle

%\keywordlist

%\conceptlist

\section{Introduction}
\label{sec:intro}


%para 1: importance of urban modeling; lots of scanning, but lack of proc. modeling
Obtaining detailed 3D urban models is important for a variety of applications ranging from urban planning and environmental simulations to virtual reality and video game creation. Given the
\begin{wrapfigure}[12]{r}{0.5\columnwidth}
\vspace*{-3ex} % 2ex is about the height of a line (twice the height of a lowercase x)
\hspace*{-0.06\columnwidth}  % figwidth - hspace = image width
  \includegraphics[width=0.56\columnwidth]{../images/teaser/inset.png}
\end{wrapfigure}
importance of such models, extensive efforts have been undertaken to create polygonal meshes from aerial images or light detection and ranging (\lidar) scans. Such  datasets are often very  expensive  and tedious to create.
They are difficult to use because they are typically heterogeneous with sparse or missing details. More importantly, they lack semantic structure, which prevents easy use in subsequent applications. 

%para 2: proc modeling is attractive. one of the most natural choices is procedural extrusion (PE) as this is naturally how buildings are created --- blocks split into parcels + profiles attached to parcel boundaries
\begin{figure*}[t!]
    \centering
  \def\svgwidth{\linewidth}  
    \input{../images/megafacades/comparision.pdf_tex}
  \caption{{\it Baseline methods.} (a)~\GISds\ represent plot ownership more accurately than building structure. (b)~Image features, such as windows (cuboids) extracted from \streetI, are available only near where the images have been taken (cubes), and lack information about the interior of the structure; different images may give contradictory features for the same building. (c)~Raw polygonal meshes tend to be more complete, but they contain noise and are typically polygon soups. One reconstruction possibility is to fit horizontal ``floors'' to the mesh~(d), while another is to extrude the GIS footprints to heights available from a database~(e). Both these approaches fail to convey the roof structures of the input. A popular GIS data visualization techniques is to create a \emph{hip roof} over all footprints (f), which leads to a monotonous structure. (g)~Naively applying profiles from the input mesh to the \GISds\ leads to more interesting roof shapes; but these are inaccurate because the GIS edges are frequently not representative of real-world building walls.}
  \label{fig:baseline_motivation}
  \vnudge
\end{figure*}


In contrast, procedural pipelines (e.g., CityEngine) create homogeneous, semantically labelled urban models. 
One such procedural pipeline uses horizontal (building) footprints and the corresponding vertical profiles to create mass models by extruding the footprint upwards along the profiles, which may then be `decorated' with building elements such as windows, doors, etc.
Currently, this workflow is suitable for coarse approximation of larger areas, or for detailed manual modeling of particular (iconic) buildings, but it does not scale to accurate detailed modeling of wider urban areas.

In this paper, we focus on the problem of \neu{procedurally} creating \outputMs by leveraging data from multiple sources (see \figref{fig:teaser} and the inset aerial view for reference).
Such raw information has different strengths and weaknesses: for example, 
publicly available \emph{Geographic Information System footprints} (\GISds) carry reliable records of plot ownership, but they often do not reflect built reality; polygonal meshes, often in the form of polygon soups obtained by processing aerial images, provide coarse information, but they lack semantic partitioning or fine details; \streetI (e.g., \facade photographs) provides detailed information, but it lacks 3D information or semantic labels. Further, each data source has its own coordinate system, suffers from distortion, and frequently contains mutually conflicting or partial information. 

%para 4: we propose inverse PE via data fusion.
Naively combining information across the above datasources results in various types of artifacts (see Figure~\ref{fig:baseline_motivation}). For example, extruding \GISds\ with profiles extracted from mesh data creates misleading mass models, while transferring window locations regressed from images onto estimated \facade planes results in poorly positioned windows.
%

Instead of heuristically combining the above datasources, we propose a unified fusion algorithm. 
%
We develop an optimization formulation that analyzes the heterogeneous data sources (i.e., \GISds, polygonal meshes, and \streetI) and retargets them to a single consistent representation. By balancing the various retargeting costs, our algorithm reaches a consensual {\em \outputM}, the output of which is building-level footprints, associated profiles along the footprint boundaries, and \facade elements placed appropriately over the  mass models (see Figure~\ref{fig:teaser}). The raw input data to our algorithm comes from various preferred layout directions (extracted from GIS information), candidate building footprints and profiles (extracted from the polygonal meshes), and \facade partitions with associated elements (extracted by analyzing the individual \facade images). Our system automatically decides {\em which} of these elements to retain and {\em how} to adapt the selected elements to create consistent output. 
%
Figure~\ref{fig:results} shows the input \GISds\ and the extracted building footprints produced by our algorithm. We note that the result is semantically structured in the sense that the output has labels associated with the different sections of the output model (e.g., windows, balconies, shops, walls, roofs, etc.).
%
Further, our algorithm does {\em not} make Manhattan-world assumptions, nor does it restrict the roof angles (i.e., roofs can be flat or sloped), nor number of pitches (i.e., \facades can alternate an arbitrary number of times between wall and roof). 

%para 5: evaluation  + summary of contributions
We demonstrate the effectiveness of our system by evaluating four differing urban settings: {\em Detroit} as a suburban US city with simple detached houses, {\em New York} with blocks of near-regular high-rise buildings arranged on a (literal) Manhattan-grid, {\em Oviedo} as a typical historic European city with non-axis aligned buildings surrounding inner courtyards, and 
{\em London} with dense urban architecture with many annexes and complex roof shapes.
%
Finally, we semantically reconstruct a very large area of central London covering 37 blocks around \LondonOC\ and compare our method with state-of-the-art urban reconstruction techniques.


In summary, we introduce a novel wide-area fusion algorithm that semantically combines multi-channel, noisy, and conflicting information to produce {\outputM}s in the form of building mass models with associated \facade elements. We demonstrate the automated method on urban neighborhoods spanning several building blocks at a scale that has not been previously demonstrated.


\section{Related Work}

We review the relevant literature on the urban modeling and reconstruction pipeline (see~\cite{musialski2013survey} for a  survey).

\subsection{Reconstructing mass models}
There are multiple possible inputs for large-scale urban mass modeling. Mass models are often reconstructed from aerial images or \lidar~\cite{brenner2005building}. Other modalities, such as synthetic aperture radar~(SAR), ground based photographs, or videos, are less common. Furthermore, satellite data have lower resolution and drones can capture only smaller areas.
While \lidar produces point clouds directly, images must be processed to produce sparse~\cite{snavely2006photo} or dense~\cite{furukawa2010accurate,cmzp_symmCalib_tog13}~point clouds. 
Some integrated modeling pipelines extract mass models from images directly~\cite{dick2004modelling,vanegas2010building,garcia2013automatic}. 
Surface models can be extracted from point clouds, e.g.,  by resampling onto a grid~\cite{poullis2009automatic}, 2.5D contouring~\cite{zhou20102}, relation-based primitive fitting~\cite{monszpart2015rapter}, or Poisson reconstruction~\cite{kazhdan2013screened}.
Another important component in urban modeling is segmentation~\cite[e.g.,][]{matei2008building,golovinskiy2009shape,verdie2015lod} to separate buildings from other classes.

Our work is mainly related to shape abstraction and simplification; we aim to create simple and plausible mass models from noisy input data. One simple model for shape abstraction is to regularize the models using the Manhattan-world assumption~\cite{li2016manhattan}. Alternately, very good results can be achieved by fitting parametric building blocks to height fields~\cite{lafarge2010structural} or \lidar input~\cite{lin2013semantic}, exploiting non-local regularity relations~\cite{zheng2010non}, or obtaining depth-layer relations by jointly analyzing images and \lidar scans~\cite{li20112d}. 
Following Verdie et at.~\shortcite{verdie2015lod}, we use a noisy building mesh as input. They use a simplified version of Globfit~\cite{li2011globfit} to detect relationships between extracted planes to regularize the output. In contrast to this method, we  jointly analyze the different input data modalities to produce a consistent \outputM, in which, for example, the footprints of the mass models are in agreement with how the \streetI\ is partitioned into different buildings. 

\subsection{Fa\c{c}ade parsing}
% 
The goal of \facade parsing is to extract \facade elements such as windows, doors, and balconies. The input of \facade parsing is typically a single image or a point cloud.
A typical initial step of \facade parsing is to compute local per-pixel information, such as segmentation information~\cite{martinovic2012three}, edge detection, or symmetry detection~\cite{muller2007image}. This  input is then regularized to make it more compliant with a given model of a \neu{\facade structure~\cite{cohen2014efficient}}.
One possible model is a grid with one spacing parameter for each row and each column~\cite{muller2007image},  which can also be represented by a rank-one matrix~\cite{yang2012parsing}.
A more general model is a hierarchical splitting tree, in which each internal node splits into multiple horizontal or vertical slices~\cite{shen2011adaptive,dai2012learning,riemenschneider2012irregular,teboul2013parsing,kozinski2015mrf}. These hierarchical approaches differ in how they incorporate low-level features stemming from  classifiers and in how they use encoded architectural knowledge. Example solutions include use of MRFs~\cite{kozinski2015mrf}, extending the CYK algorithm~\cite{riemenschneider2012irregular},  application of reinforcement learning~\cite{teboul2013parsing}, post-processing by optimization~\cite{martinovic2012three,nan2015template,jiang2016automatic}, or jointly optimizing for template matching and deformation estimation~\cite{duygu:16:cgf}.
A significant simplification used by these systems is to consider only \facade images that have been rectified and cropped for individual buildings.

\pagebreak

\subsection{Interactive reconstruction}
To achieve improved results, another line of work investigates interactive techniques for mass modeling~\cite{debevec1996modeling}, or \facade parsing. For example, Nan et al. present an interactive \facade modeling system for \lidar data~\shortcite{nan2010smartboxes} and Xiao et al. propose an interactive system for images~\shortcite{xiao2008image}. Another recent concept is to train multiple neural networks to interactively create procedural models from input sketches~\cite{nishida2016interactive}. In contrast, we aim to create an automatic system.

In this work, we build on the geometry of the \neu{straight skeleton~\cite{aichholzer1996novel}} to model architecture. Early work used the \emph{unweighted} straight skeleton to \neu{model roofs~\cite{laycock2003automatically, muller2006procedural}} and walls ~\cite{fang2013image}. 
The \emph{weighted} skeleton~\cite{eppstein1999raising} offered enhanced expressiveness; in particular, the \emph{procedural extrusion} system (PE)~\cite{kelly2011interactive} consisted of stacked weighted skeletons. \neu{Recently, Biedl et al.~\shortcite{biedl2016planar} reinforced the theoretical underpinnings of the weighted straight skeleton, renewing our interest in PEs.} Essentially, PEs are a parameterization of architecture into a horizontal 2D plan with a set of vertical 2D profiles that are associated with the edges of this plan. Such a parameterization can represent buildings with arbitrarily angled walls and roofs to provide a strong architectural prior. In this work, we develop a method to project real-world data into the space of buildings represented by PEs.




\section{Problem Setup}
\label{sec:setup}



\begin{figure}[t!]
    \centering
  \def\svgwidth{1\columnwidth}  
    \input{../images/megafacades/overview2.pdf_tex}
    \caption{{\it Overview.} Starting from \GISds, a coarse 3D mesh, and \streetI, we extract a set of \sweepedges, $\mathcal{S}$, a set of \cleanprofiles, $\mathcal{C}$, and a set of \buildingfacades, $\mathcal{B}$. These are then globally optimized to produce a semantically parsed building block as output.}
    \label{fig:setup}
    \vspace*{-.07in}
\end{figure}

Our system takes input from three sources --- publicly available \GISds, a coarse 3D mesh, and street-level \facade\ images --- with the goal of reconstructing a high-quality semantic model of an urban area. 
%
Since the different input sources have complementary strengths and weaknesses, we first process them individually to extract three types of entities: {\em \sweepedges}, {\em \cleanprofiles}, and {\em \buildingfacades}. 
%
In the following, we describe these entities, while deferring the details of how they are computed to Section~\ref{sec:processing}; the global optimization, which fuses them to produce the \outputM, is discussed in Section~\ref{sec:globopt}. Figure~\ref{fig:setup} presents an overview of our framework. 

\subsection{\GISds} 
Typically, an urban building block consists of several densely packed buildings (up to 100 buildings in our examples). 
While \GISds~(see~\cite{osm}) provide an accurate ownership record, 
surprisingly they provide little usable information concerning a building's physical walls and partitions, making it challenging to use these data directly for reconstruction. However, we found that they carry a mixture of accurate and noisy orientation information, which we utilize to regularize the processing of other data sources.


\subsection{Coarse 3D mesh} 
A 3D mesh or polygon soup (e.g., obtained via multi-view stereo or \lidar scans) provides approximate, incomplete, noisy, \neu{but large-scale} geometric information. 
We process such meshes to produce two entities: horizontal \emph{\sweepedges} and vertical \emph{\cleanprofiles} (see Figure~\ref{fig:terminology}); such \sweepedges are extruded along \cleanprofiles to create a mass model.
% 
Specifically, we extract a set of lines, referred to as \sweepedges, $\mathcal{S}$, on the ground-plane by identifying likely \facades over the mesh. 
Along these \sweepedges, we vertically slice the mesh to create many \emph{\rawprofiles}; these are clustered, averaged, and abstracted to create a \emph{set} of \emph{\cleanprofiles}, $\mathcal{C}$ (see Figure~\ref{fig:terminology} and Section~\ref{sec:profiles}).
%
Direct reconstruction from these \sweepedges and \cleanprofiles is challenging as PEs require watertight \footprintpolygons, with a \cleanprofile assigned to each edge. Specifically, there are two sources of difficulty: the \sweepedges have gaps, may self-intersect, or even be missing entirely in regions, while the \cleanprofiles are the output of local analysis, thus lacking information about building partitions and containing different sources of noise (e.g., from initial reconstruction, trees, or vehicles).

\begin{figure}[t]
    \centering
     \def\svgwidth{1\columnwidth}  
    \input{../images/megafacades/term.pdf_tex}
    \caption{\neu{{\it Terminology.} Left: The data are used to create \streetI with associated \facade planes (orange), \rawprofiles (blue) and \sweepedges (pink). Center: These are processed to create the input to the optimization --- a smaller set of \cleanprofiles (blue), \buildingfacades (orange lines), and \buildingfacadepoints (orange points), and a \groundplane tessellation consisting of \sweepedges (pink) and \softedges (black) enclosing \faces. Right: The output of the optimization is a collection of watertight \footprintpolygons (pink and purple), with a \cleanprofile assigned to every edge, and positions for every \buildingfacade (orange).}}
    \label{fig:terminology}
    %\vnudge
\end{figure}


\subsection{Street-level \facade\  images}  
Complementary to the above data sources, street-level imagery provides  information over portions of the urban blocks. Such images typically come with estimates of camera position and orientation.
For each image, we use a convolutional neural network (CNN) based supervised classifier (see Section~\ref{sec:cnns}) to detect the rectangular bounds of a \facade as well as elements such as windows, doors, and balconies. We refer to this rectangular \facade containing a collection of extracted elements as a \emph{\buildingfacade} (see Figure~\ref{fig:terminology}). Each side of a city block will typically consist of multiple overlapping \buildingfacades: one from each of the images. However, such raw \buildingfacades, $\mathcal{B}$, may contain position and orientation errors, have inconsistent scales, sometimes overlap, or be incomplete (e.g., occluded by trees, vehicles, or scaffolding). \neu{The \groundplane location of the observed start or end of a \buildingfacade in the \streetI is referred to as a \emph{\buildingfacadepoint}.}

These three data sources are in three different coordinate systems, and may introduce conflicting information, making their combination challenging.  
Further, each is subject to reprojection and inherent noise, both within and between datasets. For example, we found that the given location and orientation of \buildingfacades varied on different sides of a building due to GPS or GIS errors. Poor correlation between the image and 3D mesh was sometimes observed because of differing scale estimates or changes in the environment (e.g., buildings had been constructed, modified, or demolished). 


%%%%%%%%%%%%

\subsection{Notation}
\label{sec:notation}
%
Before we formulate the main binary integer program~(BIP) that processes these inputs, we first introduce some notation. We use \sweepedges, $\mathcal{S}$, to oversegment the \groundplane ($y=0$) to form a tessellation of faces, $\mathbb{G}$, as described in Section~\ref{subsec:formulation}. 
%
Our algorithm determines whether or not each edge, $\edge_k \in \mathbb{G}$,  should be selected, thus implicitly encoding the final building \footprintpolygons. We represent this selection with a binary indicator variable,  $\isEdge^k$, such that $\isEdge^k=1$ if the edge, $\edge_k$, is selected and forms part of a \footprintpolygon,\sout{ (selected edges are referred to as \emph{\planedges},} and $\isEdge^k=0$ otherwise. 
%
Note that in densely built urban areas, even though adjacent buildings can share a common wall, the structures often have different heights or roofs.
We encode such a situation by  two, possibly different, profiles associated with the two sides of each interior wall, $\edge^k$. (For the remainder of the paper, we discuss one such profile per edge, while the other one is similarly treated.)  We denote the length of any edge, $\edge_k$, as $\|\edge_k\|$  and the
maximum mesh height above a point on the \groundplane, $(x,z) \in \mathbb{R}^2$, 
as 
$\height{}(x,z).$


We use logic operators (such as $\land, \lor, \oplus, \neg$) noting that each can be expressed in BIP constraints with additional variables (detailed in Appendix A).
We will not explicitly introduce such extra variables and constraints, but we use the logic operator directly.

Unlike $\isEdge^k$, which is an individual binary variable, we will have cause to represent categorical variables (such as color or profile choice) using \emph{selection vectors}. Note they are also called `one hot vectors' in the literature.
We denote a selection vector of length $n$ as $\mathbold{\chi}:=(\chi_1,\dots,\chi_n)$; each element (such as $\chi_1$) is a binary variable. Selection vectors have {\em exactly} one element set to one, while the others are all zero. We encode this condition  with the constraint
\sout{by requiring each bit $\chi_i\in\{0,1\}$ and the \sout{side }constraint:}
$
\sum_{i=1}^n{\chi_i} = 1.
$
We will wish to compare two selection vectors. For example, given $\mathbold{\chi}:=(\chi_1...\chi_n)$ and $\mathbold{\psi}:=(\psi_1...\psi_n)$, we desire an output of $0$ if all elements are equal (i.e., $\chi_i = \psi_i, \; \forall i$), and $1$ otherwise. To simplify notation in this situation, we  write
\isDifferent{\mathbold{\chi}, \mathbold{\psi}}\ to indicate 
$$
\isDifferent{\mathbold{\chi},
\mathbold{\psi}} = (\chi_1 \oplus \psi_1)\lor \dots \lor (\chi_n \oplus \psi_n). 
$$
Note that the above macro describes a set of variables and constraints to be added to the BIP. 




\section{Fusion  Optimization}
\label{sec:globopt}





So far, we have introduced: 
(i)~a set of \sweepedges, $\mathcal{S}$ (for extraction details see Section~\ref{sec:profiles}); 
(ii)~a set of \cleanprofiles, $\mathcal{C}$ (Section~\ref{sec:profiles}); and 
(iii)~a set of \buildingfacades, $\mathcal{B}$ (Section~\ref{sec:cnns}).
We continue to formulate a global optimization that fuses these entities to output a semantically parsed building block, simply referred to as the \mbox{\em \outputM} (see Figure~\ref{fig:setup}). 

To achieve this, we address three key challenges: 
(i)~identifying \textit{\footprintpolygons} for each building in the \groundplane tessellation;
(ii)~selecting a \cleanprofile from $\mathcal{C}$ for each edge of every \footprintpolygon; and 
(iii)~retargeting \buildingfacades from $\mathcal{B}$ to a subset of the edges of the \footprintpolygons. A good \buildingfacade location matches the mass models that are implicitly obtained by extruding the \footprintpolygons along the selected \cleanprofiles.

 
Note that the above problems are tightly linked and must be solved together. For example, the boundary of a \footprintpolygon depends on which profiles are selected, which in turn depends on how the \buildingfacades are retargeted to match 3D mass model boundaries.


\begin{figure}[t!]
    \centering
  \def\svgwidth{\columnwidth}  
    \input{../images/megafacades/subdiv.pdf_tex}
  \caption{{\it \Sweepedges and \softedges.} A set of \sweepedges (a, pink) are extended to oversegment the \groundplane~(b) into faces. The \sweepedges are inserted one at a time, in order of decreasing length. To complete the tessellation, the \sweepedges are extended by \emph{\softedges}~(black). The \buildingfacadepoints~(c) further subdivide the \groundplane if there are no existing similar edges. Finally, we remove faces that are mostly outside the GIS footprint (d, green) to create the tessellation, $\mathbb{G}$.}
  \label{fig:subdiv}
  \vnudge
\end{figure}


\subsection{Formulation}
\label{subsec:formulation}
We simultaneously address the above challenges by formulating a BIP; we next describe the optimization variables, constraints, and objective terms associated with each challenge.

\subsubsection{Identifying \footprintpolygons} The input \GISds, \streetI, and 3D mesh carry noisy and incomplete information about individual buildings. This is particularly pronounced in densely built urban areas where adjacent buildings often share walls, contain courtyards, and regularly break the Manhattan-world assumption. Using the available information, we first oversegment \neu{the \groundplane into \emph{\faces} using the \sweepedges}, then merge the oversegmented regions, and finally extract the \footprintpolygons. %We start by describing the oversegmentation process. 




First, we extend the \sweepedges in $\mathcal{S}$ to initiate the \groundplane oversegmentation (see Figure~\ref{fig:subdiv}a).
%
Note that only the edges created by \sweepedges have profiles, while others, called {\em \softedges}, complete the tessellation (see Figure~\ref{fig:subdiv}b).
%
Next, we use the estimated \buildingfacadepoints (shown as blue dots in Figure~\ref{fig:subdiv}c) from the \streetI\ to further oversegment the ground plane by adding \softedges that are perpendicular to the \buildingfacade 
into the tessellation. All these edges indicate potential separating walls between adjacent buildings.
%
Finally, we discard faces that are either mostly outside the \GISds, or have a mean mesh height below a threshold (3m in our data). 
We use $\mathbb{G}$ to denote the resulting tessellation (see Figure~\ref{fig:subdiv}d).



 \begin{figure}[t!]
    \centering
    \def\svgwidth{\columnwidth}  
    \input{../images/megafacades/partition.pdf_tex}
    \caption{{\it Oversegmenting the \groundplane.}  We use \sweepedges and \GISds\ to overpartition the ground plane. 
    Left:~The \sweepedges~(pink) along with their \softedge extensions~(black) partition the plane. Center:~Further oversegmentation based on the building-\facades extracted from \streetI\ (blue). Right:~using height and GIS information (green) we identify the interior \faces to produce the oversegmentation, $\mathbb{G}$.} 
    \label{fig:partition}
\end{figure}

% 
Extracting \footprintpolygons amounts to setting the BIP variables, $\isEdge^k$, for each of the edges, $\edge_k$, surrounding every \face, $f_i \in \mathbb{G}$.
%
However, setting up such an optimization is cumbersome, as not all values for $\{\isEdge^k\}$ result in valid partitions of the ground plane (see Figure~\ref{fig:valid}). Hence, we indirectly formulate the problem by deciding which neighboring \faces in the tessellation $\mathbb{G}$ should be merged to produce the final building \footprintpolygons.
%
For example, the resulting tessellation for Figure~\ref{fig:teaser} is shown in Figure~\ref{fig:partition}. 



\begin{figure}[b!]
    \centering
  \def\svgwidth{0.8\columnwidth}  
    \input{../images/megafacades/valid.pdf_tex}
  \caption{\textit{Valid \footprintpolygons.} \neu{Left: A set of edges, $\{\isEdge{}\}$. Center: Two geometrically invalid partitions using those edges caused by self-grazing polygons (a), dangling edges (b), and holes in the boundary (c). Right: Valid \footprintpolygons are map-coloring solutions.}}
  \label{fig:valid}
\end{figure}

%
%
%

The \footprintpolygons should ideally follow the \sweepedges, while making them watertight, and should use as few \softedges as possible to fill in sections of missing data. Further, we  encourage selection of edges where there is a large height difference on either side of a sweep-edge \neu{} (e.g., between adjacent buildings). 
%
For each such \face $f_i \in \mathbb{G}$, we sample $\height{}(x,z)$ using the mesh data to find the mean height over the \face, $\height{}(f_i)$. This averaging adds robustness over problematic mesh features such as holes. The height difference across an edge is thus $\te{heightDiff}(\edge_k)  = |\height(f_i)-\height(f_j)|$  where $f_i$ and $f_j$ are the \faces incident to $\edge_k$.


\pagebreak

{\noindent \em Selection variables:} % 
The \face-merging problem can be reduced to a region- (or map-) coloring problem with adjacent \faces of the same color indicating that the \faces are implicitly merged. 
Thus, for each $f_i$, we assign a selection variable, $\mathbold{\gamma}^i$, with length 5. Although four colors are sufficient for map-coloring, we found experimentally that our BIP converges faster with an extra color.


{\noindent \em Constraints:} 
The edge-selection variable, $\isEdge^k$, defines if an edge, $e_k$, lies on a \footprintpolygon; usually this is because it lies between \faces of different colors.
% 
Thus, for all edges, $\edge_{k}$, between two faces $f_i$ and $f_j$, we require
\begin{equation*}
\isEdge^k = \isDifferent{\mathbold{\gamma}^i,\mathbold{\gamma}^j},    
\end{equation*}
which amounts to a set of variables and constraints as introduced in Section~\ref{sec:notation}.
%
Since all other edges, $\edge_k$, are at the boundary and must be part of a \footprintpolygon, we set their
$\isEdge^k$ to $1$. 



{\noindent \em Objective terms:} 
In formulating the selection of edges from the tessellation, $\mathbb{G}$, we add penalties for the following conditions: 
($O_1$)~if a \sweepedge \emph{is not} selected or a \softedge \emph{is} selected; and 
($O_2$)~if an edge with high height differential is {\em not} selected

\begin{align*}
O_1( \{\isEdge^k\} ) &:=& 
  &\sum_{\edge_k \in \mathbb{G}} 2 \|\edge_k\| (\neg\isEdge^k \land \hasProfileEdge{\edge_k}) \nonumber \\ 
&&+ &\sum_{\edge_k \in \mathbb{G}}
\|\edge_k\| (\isEdge^k \land \neg \hasProfileEdge{\edge_k}) \nonumber \\ 
O_2( \{\isEdge^k\} ) &:=& &\sum_{\edge_k \in \mathbb{G}}  \|\edge_k\|  \; \te{heightDiff}(\edge_k) \neg \isEdge{}^k,  \nonumber
\end{align*}
where  
$\hasProfileEdge{\edge_k}$ returns 1 if the edge, $\edge_k$, is a \sweepedge, or 0 if it is a \softedge.

\subsubsection{Selecting \cleanprofiles} 
%
The input mesh data are noisy, incomplete, and often contain spurious geometry (e.g., trees or cars). Our goal is to abstract the raw input by assigning a \cleanprofile from the set, $\mathcal{C}$, to every $\edge \in \mathbb{G}$. These assigned profiles guide the \footprintpolygon extrusion, implicitly producing a clean and abstracted PE mass model. 

Ideally, above each edge, the selected profile closely approximates the mesh geometry. Further, due to stability considerations when modeling with PEs, it is important that edges from adjacent and nearly parallel edges in the same \footprintpolygon select the same profile (see Figure~\ref{fig:nearParallelProfiles}). Note that this caveat does not require buildings to conform to the Manhattan-world assumption.



{\noindent \em Selection variables:} 
For every edge, $\edge_k$, we create a profile selection vector, $\mathbold{\eta}^k$, to indicate which \cleanprofile is selected from the global set, $\mathcal{C}$. The length of this vector is the size of the profile set, $\mathcal{C}$, typically 4-80 profiles.


{\noindent \em Constraints:} 
We wish \cleanprofile selections to be equal for parallel adjacent edges within the same \footprintpolygon. In other words, two adjacent edges that are nearly parallel can select different profiles {\em only if} the they belong to different \footprintpolygons --- i.e., there is at least one separating wall between them. 

Thus, for all vertices of the tesselation, $\mathbb{G}$, we create an auxiliary variable for each pair of adjacent and approximately parallel (we use a tolerance of $0.1$ radians) edges, $\edge_j$ and $\edge_k$, as
%
$$\differentProfiles{}^{(j,k)} =
\isDifferent{\mathbold{\eta}^j,\mathbold{\eta}^k}.$$
%
Because we allow only parallel and adjacent edges to have different profiles ($\differentProfiles{}^{(j,k)} = 1$) when there is at least one selected edge ($\isEdge{}^l$ = 1 for edge $\edge_l$) between them at their shared vertex (Figure~\ref{fig:nearParallelProfiles}), we require 
%
$$ \differentProfiles{}^{(j,k)} \leq \sum_{\edge_l \in \te{between}{(j,k)}} \isEdge{}^l,$$
%
where $\te{between}(j,k)$ denotes the set of edges lying between $\edge_j$ and $\edge_k$ and sharing a common vertex. 
% -- ignores the fact that "between", means on the inside angle.
We implement $\mathbb{G}$ as a half-edge data structure, which permits direct implementation of the $\te{between}()$ operator.

\begin{figure}[t!]
    \centering
  \def\svgwidth{1\columnwidth}  
    \input{../images/megafacades/optim2.pdf_tex}
  \caption{{\it Undesirable \facade\ splits.} Left-center: PEs are unstable when different profiles (blue) are selected on nearly parallel edges (green); moving a single point (orange) a short distance creates a very different result.  Right: To avoid this situation, the \cleanprofiles of the adjacent parallel edges (given by the selection vectors $\mathbold{\eta}^j$ and $\mathbold{\eta}^k$) are constrained to be equal, if the dividing edge is selected ($\isEdge^l = 1$).}
  \label{fig:nearParallelProfiles}
  \vnudge
\end{figure}


{\noindent \em Objective term:}
For each edge, $\edge_k$, let the corresponding set of \rawprofiles obtained by vertically slicing the input mesh be $\mathcal{R}(\edge_k)$.
%
Let the vector $\mathbf{F}_k$ list the error in fitting each \cleanprofile, $p_c \in \mathcal{C}$, to all the \rawprofiles, $q \in \mathcal{R}(\edge_k)$, along the edge, $\edge_k$. This error is measured by the function $d()$, which measures the difference between two profiles (see Section~\ref{sec:profiles} for details). 
% 
Specifically, each element of the vector, $F^c_k$, is computed for a single \cleanprofile, $p_c \in \mathcal{C}$, over all the edge's \rawprofiles as
% 
$$
F^c_k = \sum_{q \in \mathcal{R}(\edge_k)}{ d(p_c, q, {\min}_{\te{Y}}(q), {\max}_{\te{Y}}(q))}.
$$
Note that for the above computation, $p_c$ is moved to align with $q$ at height $y=0$ (i.e., on the \sweepedge). Further, the function $d()$ is evaluated over the \rawprofile's height, $[{\min}_{\te{Y}}(q),{\max}_{\te{Y}}(q)]$, to match \rawprofiles with ends at varying heights to the more complete \cleanprofile.
%
If there is no \rawprofile associated with an edge, we set the assignment cost vector, $\mathbf{F}_k$ to $[-1,0,\dots0]$, i.e., we give a small bonus to selecting the vertical \cleanprofile. (Note that the -1 favors the default vertical profile in the absence other information.) 
%
%
We can now define an objective term for each edge, $\edge_k$, measuring the fit of the selected \cleanprofile to the supporting edge's \rawprofiles,
$$ O_3 ( \{ \mathbold{\eta}^k \} ) :=  \sum_{\edge_k \in \mathbb{G}}   \|\edge_k\| \mathbf{F}_k \cdot   \mathbold{\eta}^k.$$ 
%
%
We recall that each internal \sweepedge potentially has two sets (for a shared wall) of \rawprofiles associated with it, corresponding to the two adjacent buildings. The above cost is adapted accordingly when a pair of edges is present. 




\subsubsection{Retargeting \buildingfacades} 
%
\StreetI of \facades contains valuable information about building placement. For example, neighboring buildings may have different materials which provides evidence about their widths, or a change in \facade height may advocate splitting a \footprintpolygon. However, \streetI often does not align with the 3D mesh (or even other images) --- both in position and scale. We extend our formulation to include such \streetI by observing that solving for alignment and scaling is equivalent to establishing correspondence between the start and end \buildingfacadepoints, and the vertices on the boundary of the tessellation. 

\neu{Specifically, let the set of vertices on the outer boundary of $\mathbb{G}$ be $\mathbb{V}$. We aim to assign every \buildingfacadepoint to a vertex, $v \in \mathbb{V}$.} Because the error in the \buildingfacade location is of a known maximum distance (approximately 3m in our datasets), we can enumerate the nearby boundary vertices for each \buildingfacadepoint. In the process, we aim to minimize both the \buildingfacadepoint displacement and the height disparity between the \buildingfacade-based (\streetI), and mesh-based, estimates. We note that multiple images may create overlapping \buildingfacades, with each suggesting a corresponding set of \facade elements.

{\noindent \em Selection variable:} 
We cluster nearby \buildingfacadepoints to a group, $\cluster_i$, with a cluster-representative denoted by $m^i_\star$. 
For each cluster-representative, we find the nearby boundary vertices in $\mathbb{V}$, denoted as $\te{nearby}(m^i_\star)$. We use a selection variable, ${\tau^{(i,w)}}$, to identify the points in $\cluster_i$ mapped to vertex $v_w$. 


{\noindent \em Objective terms:} %
We introduce three terms: 
($O_4$)~to discourage stretch and height disparities between heights extracted from the mesh and those from the \streetI; 
($O_5$)~to encourage \buildingfacadepoints to pick exterior corners of the tessellation; and 
($O_6$)~to reduce splitting of \footprintpolygons under a \buildingfacade. 

First, to minimize stretch and height disparity of the \buildingfacades (see Figure~\ref{fig:mfpoint}), we add
% 
\begin{multline*}
O_4 ( \{ \tau^{(i,w)} \} ) :=\\
\sum_{\forall \cluster_i} 
\sum_{m_a \in \cluster_i} 
\sum_{w \in \te{nearby}(m^i_\star) }
\tau^{(i,w)} ( \distance{}{}(v_w, m_a) +\\
| \htLeft{}( m_a ) - \htLeft{} ( v_w ) | +\\
| \htRight{}( m_a ) - \htRight{} ( v_w )| ),
\end{multline*}
% 
where the function $\distance{}()$ gives the distance between a boundary vertex and \buildingfacadepoint, and $\htLeft{}()$ gives the \buildingfacade height or face height  (from the \streetI or the 3D mesh, respectively), on the left (similarly for $\htRight{}()$), as shown in Figure~\ref{fig:mfpoint}. 
% 


\begin{figure}[t]
    \centering
  \def\svgwidth{\columnwidth}  
    \input{../images/megafacades/mfpoint.pdf_tex}
    \caption{{\it Stretch and height disparity.} Left: We evaluate the fit of the \buildingfacade(blue) to the 3D mesh (grey) using stretch and height disparity. Right: The \buildingfacadepoints, $m_1, m_2, m_3$, are grouped into a cluster, \neu{$\mathbb{C}_i$}, with representative $m^i_\star$. The indicator variable $\gamma^{(i,1)}$ (or $\gamma^{(i,2)}$) denotes which of the points in cluster \neu{$\mathbb{C}_i$} are mapped to the boundary vertex, $v_1$ (or $v_2$).}
  \label{fig:mfpoint}
\end{figure}



It is particularly desirable to assign a \buildingfacadepoint to a corner vertex of the tessellation boundary (a subset of $\mathbb{V}$); thus, it receives a reward
$$
O_5 ( \{ \tau^{(i,w)} \} ) := -  \sum_{v_w \in \te{corners}} \tau^{(i,w)},
$$ 
%
where the set $\te{corner}$ contains all vertices adjacent to two boundary edges of $\mathbb{G}$ that meet at $[\pi/3,2\pi/3]$.  

%


%
\pagebreak

\begin{wrapfigure}[12]{r}{0.25\columnwidth}
%\vspace*{-.1in}
\hspace*{-.2in}
  \def\svgwidth{0.26\columnwidth}  
    \input{../images/megafacades/edgenomini.pdf_tex}
\end{wrapfigure}
Finally, it is undesirable for an edge to be selected that arrives at the tessellation boundary underneath a \buildingfacade (inset: top, blue).
Such an edge may unnecessarily split a \footprintpolygon (pink). Hence, we penalize the selection of edges, $\edge_k$, that approach vertices of the boundary with \buildingfacades, but without selecting  \buildingfacadepoints. This results in improved integration of the \facade boundaries into the mass model~(inset, bottom). 
%
Specifically, we penalize such a situation as %where an edge is selected and covered by a \buildingfacade without being assigned a \buildingfacadepoint, as
% 
$$
O_6 ( \{ l^k \}) :=  \sum_{\edge_k \in \mathbb{G}} \isEdge{}^k \land \edgeNoFacade{}^k.
$$
% 

{\noindent \em Constraints:} %
%
The auxiliary binary variable, $\edgeNoFacade{}^k$, captures whether a vertex of edge, $\edge_k$, is not assigned a \buildingfacadepoint, but is covered by a \buildingfacade,
%
\begin{eqnarray*}
  \edgeNoFacade{}^k &=& \sum_{v_w \in \te{verts}(e_k)} \left(\te{free}(v_w) \sum_{\forall \cluster_i  } \tau^{(i,w)}\right) < 1.
\end{eqnarray*}
%
  The above constraint evaluates whether an edge, $e_k$, has a boundary vertex, $v_w \in \te{verts}(e_k)$, which is covered by a \buildingfacade, but is not assigned a \buildingfacadepoint by any $\tau$. The function $\te{free}(v_w)$ returns 0 if the vertex, $v_w$, is covered by some \buildingfacade and 1 otherwise.

\subsubsection{Objective function} We find a solution that satisfies all the above constraints, while minimizing
$$
\min \sum_{i=1}^{6} \alpha_i {O_i}
$$ 
over the variables $\{ \mathbold{\gamma}^i \}$, $\{\mathbold{\eta}^k\}$, $\{\mathbold{\tau}^{(i,k)}\}$, and the associated auxiliary variables.
% 
In our results, we used $\alpha_1=10$, $\alpha_2=1$, $\alpha_3=0.01$, $\alpha_4=1$, and 
$\alpha_5 = \alpha_6 =  0.1 \sum_{\edge_k \in \mathbb{G}} \|\edge_k\|$.


\subsection{Creating the Structured Model}
\label{sec:post}

Given a \emph{solution} to the above optimization, we can generate the geometry for the final \outputM. 





Starting from the \groundplane tessellation, $\mathbb{G}$, and the region coloring $\{\mathbold{\gamma}^i\}$, we merge neighboring faces to which the solution assigns the same color. The resulting 2D polygons form the \footprintpolygons of the final mass models.
%
For every edge in each \footprintpolygon, the solution contains a \cleanprofile from the set $\mathcal{C}$, in  $\{\mathbold{\eta}^k\}$. Procedural extrusions~\cite{kelly2011interactive} lift each \footprintpolygon using the selected \cleanprofiles to create a building's mass model. 
\neu{During this extrusion, we cap the PE mesh at the average mesh height (sampled by $\height{}(x,z)$ over the \footprintpolygon) to stop runaway geometry and to create flat roofs.
  An exception is when the PE horizontal cross-section area is decreasing rapidly at this height, in which case we assume that the roof is pointed.
We cap pointed roofs at a higher level given by the average \rawprofile height around the \footprintpolygon boundary.}
We classify the surfaces obtained via PE as walls or roofs using the local normals.

The optimization solution assigns the \buildingfacades to portions of the mass-model. This correspondence between \buildingfacadepoints and vertices of the \footprintpolygons is given by $\{\mathbold{\tau}\}$; from this, we can position the \buildingfacades over the mass models.

The \buildingfacade's points are found from image features. One, or both, points may be missing because they lie outside the image. If both points are present, we translate and scale the \buildingfacade to align its \buildingfacadepoints with the corresponding footprint vertices. If only one point is present, we simply translate it to align with the found vertex. In the case of no points, the \buildingfacade is aligned using estimated \GSV pose data.

\neu{In this manner, multiple \buildingfacades can be positioned over the same section of the mass model, giving us multiple position estimates for \facade elements (doors, windows, balconies etc., see Figure~\ref{fig:overlapping_features}). Further, because these elements have been estimated from \streetI, they contain noise and omissions. }


\begin{figure}[t]
    \centering
    \includegraphics[width=\columnwidth]{../images/overlapping_features/overlapping_features.png}
    \caption{{\it Overlapping building-\facade\ elements.} An urban block is typically covered by multiple overlapping \facade images, giving repeated bounding rectangles for many elements, such as \neu{windows (turquoise), shops (pink), balconies (orange), doors (green), mouldings (dark blue), and \buildingfacade boundaries (light blue)}.
    The left-most visible facade of Figure.~\ref{fig:teaser} was reconstructed from these overlapping elements.}
    \label{fig:overlapping_features}
\end{figure}


In the following,
we explain our fusion and regularization process for window elements, while other element classes (doors and balconies, etc.) are treated similarly.
We adopt a simple mean-shift~\cite{mean-shift} approach;
\neu{at each iteration, we apply a step of $0.2 \times$ the mean-shift vector to all window rectangles for a variety of parameterizations. Namely: (i) absolute position of the left, right, top, and bottom of the rectangle (to align windows with themselves and others in a grid); (ii) width and height (to maintain the shape of the windows in subsequent iterations); and (iii) spacing between adjacent windows to the left, right, top and bottom (to encourage uniform spacing between windows).}
%
After the mean-shift has converged (we use 30 iterations), we frequently have multiple rectangles associated with each window. Such rectangles are merged if the overlap is more than 50\%; otherwise, the smallest rectangles are discarded. Element rectangles are also discarded if they occur in less than half of the \streetimages that cover them. 
%


These element rectangles are added to the mass model using simple \emph{parametric models} for each type of element, such as windows, doors, window-sills, cornices, moldings, and balconies.
These are parameterized to the found dimensions, and windows or doors are recessed into the mass model \facade. As an exception, windows that lie on a mass model surface that is classified as a roof, or between surfaces with different normals, are added as dormer windows.



Finally, we color the mass model polygons classified as \emph{wall} using the information extracted from the \streetI, \neu{and those classified as \emph{roof} using optional satellite image information}.
%
Figure~\ref{fig:teaser} shows such a resulting \outputM. 


\section{Implementation Details}
\label{sec:processing}


\subsection{Extracting Sweep-edges and Profiles}
\label{sec:profiles}

We now describe the profile analysis of the 3D mesh and \GISds.
%
First, we align the mesh with the \GISd boundary. Then,  we create and cluster \emph{\horizontallines} (Figure~\ref{fig:profs}b, c) to find the \emph{\prominentfaces} of the \block (Figure~\ref{fig:profs}d). Each such face is used to compute a {\em \sweepedge} on the \groundplane, along which we extract vertical {\em \rawprofiles} from the mesh (Figure~\ref{fig:profs}e). The profiles are processed to create a small, yet representative, set of {\em \cleanprofiles}, $\mathcal{C}$. 

First, we align the mesh to the \GISds  using the GPS position associated with the mesh. We use the \GISd boundary to discard mesh geometry more than a street-lane width away (typically 4m) from the \block of interest.

\begin{figure*}[t!]
  \def\svgwidth{2.2\columnwidth}  
  \hspace*{-5mm}\input{../images/megafacades/profiles_test.pdf_tex}
  \vspace{-1cm}
  \caption{{\it Sweep-edges and profile analysis.}   A horizontal slice of the mesh (a, orange), has polylines fitted to it (b) and is regularized by the GIS information (b, green). These are clustered from the \seedlines (c, bold lines), and the associated \prominentfaces~(d), which can be used to find the raw-profiles (e). }
  \label{fig:profs}
\end{figure*}

We found \horizontallines to be good indicators of predominant directions in architectural meshes; \neu{ they also support the strong horizontal edges that are characteristic of PEs.} To find such lines, we slice the mesh horizontally (\neu{we used 20cm intervals}), and simplify each such slice using polyline fitting (Figure ~\ref{fig:profs}a).
%; this gives a map of all important (non-horizontal) faces in the mesh (Figure ~\ref{fig:profs} a). 
Because the mesh may have holes and noise, we use the directions in the \GISds to regularize the line fitting (Figure~\ref{fig:profs}b). 
Specifically, if lines are within $20^\circ$ of the closest GIS edge, they are rotated to match the GIS line's orientation.


We now cluster the fitted \horizontallines based on their orientation to identify \prominentfaces of the \block (e.g., a south-facing wall). The seed of the cluster is the longest \horizontalline (Figure~\ref{fig:profs}c, bold). From this \seedline we progressively build the cluster by adding neighboring lines (from slices above and below) in a ``floodfill'' fashion, ensuring that each line's orientation matches that of the \seedline (\neu{within $20^\circ$}).
Such a cluster of lines defines a \emph{\prominentface} over the mesh. We continue to create \prominentfaces by taking the next longest unused \horizontalline as a seed and repeating the floodfill.
We discard any \prominentfaces that cover a small area of the mesh; we use a threshold of approximately $30m^2$, which balances preserving detail with removing noise.

The \prominentfaces are now sampled to obtain profiles (Figure ~\ref{fig:profs}d). A \emph{profile} is a weakly $y$-monotone polychain (i.e., every point is greater, or equal, in height to every preceding point). This monotonic property is required by PEs, which we observe is satisfied by a large majority of building types. 
% 
%
We continue to extract a set of \rawprofiles directly from the 3D mesh; the mesh is sliced perpendicularly to the \seedline's direction at regular intervals (\neu{20cm}). Nearly horizontal mesh faces (with a normal \neu{approximately $5^\circ$ from vertical}), or those not associated with the \prominentface are ignored.  We create a \rawprofile by traversing a portion of the slice, starting at the closest point on the slice to the \prominentface's \seedline. The traversal takes place upwards and downwards, selecting monotonic line-segments from the slice to add to the profile. It jumps over small gaps and non-monotonic sections of the slice by searching for the next point in a small locale (\neu{approximately 2m}).



\begin{figure}[b]
    \centering
  \def\svgwidth{1\columnwidth}  
    \input{../images/megafacades/raw-profs.pdf_tex}
    \caption{\textit{Raw- and clean-profiles.} Left: Each color represents a cluster of adjacent and similar \rawprofiles from Figure~\ref{fig:teaser}. Right:~A cluster of \rawprofiles~(grey) has line segments fitted to it~(purple) and is finally regularized to yield a \cleanprofile~(blue).}
    \label{fig:profile_cleanup}
  \vnudge
\end{figure}


We now use the \rawprofiles to find a smaller, yet representative, set of \cleanprofiles, $\mathcal{C}$.
%
We first cluster the \rawprofiles along each \sweepedge using profile distance. 
Given two monotone profiles, $p_i$ and $p_j$, we define the profile difference at a height, $y$, as
% 
\begin{multline*}
\vspace*{-.06in}
\delta(p_i, p_j, y) =\\
\begin{cases}
    \sqrt{(x(p_i, y) - x(p_j, y))^2 +
    4(\angle (p_i, y) - \angle (p_j,y ) )^2} \\
    \quad\quad\quad\quad \text{if } p_i \text{ and } p_j \text{ are defined at height y,}\\
    10       \quad\quad\quad \text{otherwise.}
\end{cases}
\end{multline*}
%
where $x(p_i, y)$ and  $\angle(p_i, y)$ are, respectively, the x-position and angle (in radians), of profile $p_i$ at height $y$. When the profiles range between heights $y_l$ and $y_u$, the cumulative distance function is then the mean horizontal distance between the profiles discretized over the vertical range $[y_l,y_u]$ as
% 
$$
\te{d}(p_i, p_j, y_l, y_u) :=  {\sum_{y \in [y_l,y_u]} \delta(p_i, p_j, y) } / {(y_u-y_l)}.
$$
%
%
The \rawprofiles are clustered by examining consecutive profiles along each \sweepedge, starting a new cluster whenever
$$
d(p_\te{last}, p_\te{next}, 0, \te{max}_Y(p_\te{last}, p_\te{next})) > t,
$$
%
where $t$ is a threshold value and $\max_Y(p_i, p_j)$ is the maximum height of profiles $p_i$ and $p_j$. Small clusters with fewer than five profiles are discarded. Empirically, we find that forming clusters from such contiguous portions of \sweepedges gave better results than techniques such as spectral clustering, because it prioritizes the strong spatial-correlation between adjacent \rawprofiles.
Examples of such clusters are shown in Figure~\ref{fig:profile_cleanup}-left.



To create a simplified \emph{\cleanprofile} from each cluster of \rawprofiles, we fit a set of line segments \neu{(Figure~\ref{fig:profile_cleanup}-right)}. Using strong architectural priors, we regularize these lines into a \cleanprofile. Because of the low resolution of our input meshes, we found we could aid regularization by requiring the profiles to be both vertically \emph{and} horizontally monotonic (note that PEs require only that the profiles be vertically monotonic).

We used the following rules to create the \cleanprofiles~(see Figure~\ref{fig:profile_cleanup}-right): %
(i)~lines that are nearly horizontal or vertical are snapped to these 
orientations. Near the ground, this snapping is very aggressive to mitigate the effect of occluders; 
(ii) lines that do not form part of vertically and horizontally monotonic profiles are either removed or sliced so that they do; 
(iii)~lines that are near the ground are extended to the ground; and, finally, 
(iv)~if two adjacent lines could be extended to intersect within 2m of an end of both lines, we extend the lines to this intersection.
%
We add the resulting \cleanprofile to the  profile set, $\mathcal{C}$.

A large number of \cleanprofiles in $\mathcal{C}$ are computationally expensive in the optimization stage (Section~\ref{sec:globopt}). Hence, we aggressively reduce them by: (i)~removing pairs of similar profiles from the pool using $d$ \neu{(we used $d() < 1$)}; (ii) discarding any profile that is not preferred by some cluster of raw profiles, and (iii)~replacing all simple vertical profiles with a single vertical profile at the start of $\mathcal{C}$.


\begin{wrapfigure}[7]{r}{0.5\columnwidth}
\vspace*{-.2in}
\hspace*{-.1in}
  \def\svgwidth{0.52\columnwidth}  
    \input{../images/megafacades/prof-line.pdf_tex}
\end{wrapfigure}

Finally, for each \prominentface, we compute a \emph{\sweepedge}. \Sweepedges represent potential wall positions over the \groundplane, and, along with suitable vertical \cleanprofiles, create the 3D mass models.
We find a \sweepedge by projecting the \seedline of each \prominentface onto the \groundplane (inset; orange line), and offsetting it to lie close to the start of the profiles (inset; pink line).
\neu{This offset is necessary because the found \seedline may not be on the structure's wall.}
The offset is the mean horizontal distance from the \seedline to the bottom of the \rawprofiles.
This set of \sweepedges, $\mathcal{S}$, represents the potential wall-positions.

\subsection{Acquiring Street-level Imagery}
\label{sec:acquiring-images}

We use \streetI\ from \acf{GSV} to estimate the locations of \facade elements such as windows, balconies, doors, and moldings, as well as the locations of \facade boundaries. 
Unprocessed \GSV images are $360^\circ$ panoramas including approximate pose data (position and orientation of the rig used to capture the images) that are estimated using GPS and a variety of additional techniques described by Anguelov et al.~\shortcite{anguelov2010google}. 
Based on the \GSV pose information and \GISds, we project the \GSV panorama images onto the expected \facade plane to obtain a (roughly) rectified {\em projected image}. 

These projected images are generated at a resolution of 40 pixels / meter. We crop the images to a fixed horizontal field of view of $120^\circ$. This is centered on the projection of the paranorama center onto the \facade plane. We use a fixed field of view  to avoid distortion caused by projecting the panorama at extreme angles. This results in  more than one overlapping image of each \facade and many images containing only a portion of a \facade. We note that some \facades have no \GSV images because of legal and physical constraints on photography. A typical example of missing imagery is the private courtyards found in the center of many European city blocks. Next, we describe how to find \facade features in the projected images.










\begin{figure}[t!]
  \def\svgwidth{\columnwidth}  
    \input{../images/megafacades/independant-labels-comparison.pdf_tex}
\caption{{\it Training data and \facade\ classification.}\sout{ The independent labels used to train an image segmentation model;} Top-left: The ground truth used to for the `\Facade' and `Window' labels. Remainder: The source image (top-right) used to compare a model trained to recognize a single set of disjoint labels (bottom-left) with one trained to recognize independent sets of labels for each type of feature with edge labels (bottom-right). Each model was trained for 150 Epochs. The second option leads to crisper features.}
\label{fig:edge-compare}
\end{figure}


\subsection{Analyzing Street-level Imagery}
\label{sec:cnns}

%%%%%%%%%%%%%%%%%%%%%%%%

Starting from input \streetI, our goal is to detect each \facade's location and dimension, and its building elements (e.g., windows, balconies, etc.). A \emph{\buildingfacade} records this information for one image and one estimated \facade; we refer to the set of \buildingfacades as $\mathcal{B}$.

In practice, we found the \GSV pose estimates to be insufficient to produce projected \streetI that is sufficiently aligned with \GIS data. In the example of London, we observed overlaid \GSV imagery to deviate from \GIS building footprints by nearly 3m on the \facade plane, or $5^\circ$ in \GSV panoramas. 
\neu{Therefore, a pre-processing step removes parts of the images that are unlikely to be part of a \facade and then rectifies each image. The unwanted features} are segmented and masked-out using the \emph{Bayesian SEGNET} CNN~\cite{kendall2015bayesian,badrinarayanan2017segnet}. This network was trained on urban street scenes using CamVid data~\cite{BrostowSFC:ECCV08} and then refined using CityScapes data \cite{Cordts2016Cityscapes} \neu{to identify} parts of images that are likely to have \facade features. \neu{We then} rectify based on the edges within that region using the method proposed by Affara et al.~\shortcite{affara2016large}. %All further processing is done on these  images. 

Next, we identify the \facade elements within these rectified images. We refine the probabilistic Bayesian SEGNET architecture to segment a set of labels for architectural \facade element features using the CMP Facade dataset \cite{tylecek2012cmp}, the dataset used by Affara et al.~\shortcite{affara2016large}, and an additional dataset of 800 facades that we annotated directly from \GSV images of London, Oviedo, and New York. We use this \emph{SEGNET-FACADE} model (available at: \url{https://github.com/jfemiani/facade-segmentation}) to assign per-pixel probabilities to the images for each feature class. 


\begin{figure}[t!]
%\vspace*{-.15in} 

  \def\svgwidth{\columnwidth}  
    \input{../images/megafacades/facade-cuts.pdf_tex}
\caption{{\it Finding \facade extents.} Left: We split images with multiple \facades based on the peaks of the vertical sums of the \facade `Edge' scores that are output by the segmenter (superimposed in blue over the image); \facades are split at the highest point of each interval where the projection's value is more than one standard deviation ($\sigma$) above its mean ($\mu$). Right: The integral of the detected 'Sky' label (green) is used with a threshold to identify the top of the \facade.}
\label{fig:facade-cuts} 
\end{figure}



\begin{figure*}[t!]
    \centering
    \includegraphics[width=\linewidth]{../images/results/results.png}
    \vspace*{-.1in}
    \caption{{\it Oviedo, Manhattan, and London.} City blocks from Oviedo (left, center-left), Manhattan (center-right) and London: \LondonRS\ (right, as Figure~\ref{fig:teaser}). Top: Our results. Bottom: input \GISds~(green) and optimization output floorplans (blue). See also Table~\ref{table:runtimes2}.}
    \label{fig:results}
\end{figure*}

Traditional segmentation approaches, including SEGNET, assign a single label to each pixel in an image. In contrast, we treat \facade segmentation as a number of separate labeling tasks, one for each class of \facade element (window, shop, balcony, molding, door etc.), and one for the \facade extent itself. Each task assigns one of four labels to each pixel; 
`Negative', `Positive', `Unspecified' (which is ignored), or `Edge'. The 'Edge' label is automatically  assigned to a thin region (6 pixels spanning an estimated 15cm) around the edge of each feature, with the exception of vertical \facade edges, where the `Edge' label is assigned to a wider region (15 pixels wide, spanning approximately 38cm). Using a separate `Edge' label ensures more weight is given to the training-loss in these pixels \neu{due to median frequency balancing \cite{Eigen2015-aj}}. 
Empirically, these improvements result in sharper features, as shown in \figref{fig:edge-compare}, which is useful for isolating individual feature instances. The CNN processes images at a resolution of $512 \times 512$ pixels. We rescale all images to a height of $512$ pixels and crop the widths. During inference, several horizontal tiles are used to cover an image.



The \GSV images often contain multiple \facades, and it is important to separate them into different individual \buildingfacades for the optimization. 
At the inference stage, we sum each pixel column's Bayesian SEGNET-FACADE scores \neu{for the `Edge' label}. This one-dimensional signal peaks at each \facade boundary. The signal is dilated by 60 pixels (1.5m) in order to merge the dual-peaks that can occur if the \streetI is imperfectly rectified, \neu{or if there are stitching artifacts (see \figref{fig:facade-cuts})}. We extract peaks as local maxima that are more than one standard deviation above the mean of the dilated signal (see Figure \ref{fig:facade-cuts}-left). Each \facade image is split at these peaks to produce \emph{\buildingfacades}. For each \buildingfacade, we produce axis-aligned bounding boxes of all features as shown in \figref{fig:facade-cuts}-right.
%
In order to estimate the height of each \buildingfacade, we use \neu{the original} SEGNET to label pixels as `Sky'. The $85^{\text{th}}$ percentile of the scores at each pixel-row forms a one dimensional sky signal (see the green region of \figref{fig:facade-cuts}-right). The top of the \facade is the lowest point where the sky signal crosses $50\%$. \neu{These width and height estimates are assigned to each \buildingfacade and used in the optimization stage. Because we know the location of the \facade image-plane in $\mathbb{R}^3$, the \buildingfacade has an estimated 3D position, as do the associated features.}


\neu{
\subsubsection{Training and Evaluation}
We trained SEGNET-FACADE on $80\%$ (1173 images) of the data we collected, an additional $20\%$~(293 images) were used to evaluate the precision, recall, and $F_1$-scores of our approach.  
SEGNET-FACADE obtained a per-pixel precision of $96\%$, recall of $69\%$, and an $F_1$ of $0.80$.
By comparison SEGNET trained on the same data obtained a per-pixel precision of $73\%$, recall of $62\%$ and an $F_1$ of $0.67$.
We also evaluated per-object precision by defining a successful match between objects as an intersection-over-union over $50\%$. The per-object scores gave a precision of $88\%$, recall of $68\%$, and an $F_1$ score of $0.77$. We consider these to be useful results as many of the \facade images were collected ``in the wild'' from \ac{GSV} and imperfectly rectified. In comparison, SEGNET acheived precision of $36\%$ and recall of $28\%$, with an $F_1$ of $0.32$. The recent method of Affara et al. \shortcite{affara2016large} had a per-object precision of $85\%$, recall of $52\%$, and an $F_1$ score of $0.64$ on the same data. 
}


\subsubsection{Collecting color estimates} 
Although a \facade may contain a variety of texture and color patterns,  we limit ourselves to a single color; additional color variation comes from the inclusion of \facade elements with fixed colors, such as windows, molding, cornices, sills, and balconies. To estimate the color of the walls, we mask out all regions that have been identified as any other feature and estimate the mode color in the remaining pixels. Specifically, we use the \textit{Lab} color space and select $50$ colors randomly from the (unmasked) \facade. The color with the most matches is selected as representative of the \facade. \neu{Optionally, a separate color can be used for the ground floor and for the higher stories. In this case, we estimate the ground floor height by finding the highest row in the image with the `Shop' label.  }





%%%%%%%%%%%%%%%%%%%%%%%%%%%





\begin{figure*}[t!]
    \centering
    \def\svgwidth{\linewidth}  
    \input{../images/megafacades/london.pdf_tex}
    \vspace*{-.1in}
    \caption{{\it London: \LondonOC\ blocks.} Structured urban reconstruction spanning 37 blocks and 1,011 buildings. }
    \label{fig:results_london_big}
    \vspace{1cm}
\end{figure*}

\section{Results}
\label{sec:results}


\begin{figure}[b!]
  \centering
  \vspace{-0.6cm}
    \includegraphics[width=\columnwidth]{../images/detroit/detroit.jpg}
     %\vspace{0.cm}
    \caption{
    {\it Detroit}. \neu{Without building-\facades, our technique exhibits strong architectural regularization with the coarse mesh (pink) and \GISd\ (green, no interior edges) as inputs. See Table~\ref{table:runtimes1} for details.}}
    \label{fig:detroitResults}
     \vspace{0.6cm}
    %\vnudge
\end{figure}

We implemented the proposed framework using Java and Python; the sourcecode is available online at the project page (\url{http://geometry.cs.ucl.ac.uk/projects/2017/bigsur}). We used Gurobi~\cite{gurobi} for binary integer programming and Caffe~\cite{jia2014caffe} for the CNN-based classification. The timings were recorded on an i7-7700K desktop \neu{(with the exception of the Oxford Circus example)}.

\begin{table}[b!]
\begin{center}
\caption{Details for Figure~\ref{fig:results}. 
Values are given for location, number of clean profiles ($|\mathcal{C}|$) and sweep edges ($|\mathcal{S}|$), binary variables (vars) and constraints (constr), number of output footprints (fp), and the solve times.}
\label{table:runtimes2}
\begin{tabular}{|c|c|c|c|c|c|c|c|} 
  \hline
  \small
 Fig:col & location & $|\mathcal{C}|$ & $|\mathcal{S}|$ & vars & constr & fp & solve\\
                    & (lat,long) &     &  &          &  & out & time \\
 \hline
 \hline
 \ref{fig:results}:1 & 43.36635, & 75 & 61 & 32,242 &73,193&34& 15h \\ 
   & -5.83256  &    &    &       &&&     \\
 \ref{fig:results}:2 & 43.36584, & 73 & 56 & 74,694 & 148,945 &38& 5h \\ 
 & -5.83189  &    &    &       &&&     \\
 \ref{fig:results}:3 & 40.72191, & 46 & 30 & 23,172 & 49,941 & 37& 4h \\ 
 & -74.00131  &    &    &       &&&   \\
 \ref{fig:teaser}:1 & 51.51724, & 58 & 60 & 45,249 &88,171&28& 4h \\ 
\ref{fig:results}:4 & -0.14199  &    &    &       &&&     \\
 \hline
 
 \end{tabular}
\end{center}
\end{table}

We demonstrate our framework on building blocks from different cities: Detroit (see Figure~\ref{fig:detroitResults}), Manhattan and Oviedo  (see Figure~\ref{fig:results}), and London (see Figure~\ref{fig:teaser} for \LondonRS\ and Figure~\ref{fig:results_london_big} for \LondonOC). We selected building blocks to show a variety of inputs, from free standing single-family houses in Detroit to dense urban areas in the other three selected cities.
\neu{We selected cities with accessible mesh and GIS data. \sout{GSV images were sometimes inconsistent - despite the wide coverage, there were problematic areas. For example, facades were occluded by traffic, vegetation, graffiti, and murals, or had too many unusual structures such as churches, cathedrals, window blinds, and other unorthodox window decorations. Such generalization issues are to be expected from any method which learns from limited data.}
In our experiments, we found most of the parameters to be stable when the input data quality remained consistent. Typically, we adjusted two parameters before running the optimization: the thresholds for the creation of $\mathbb{G}$ and the mesh area for ignoring small clusters of horizontal lines. These parameter adjustments are relatively interactive because they occur before the slow BIP optimization.
}


\subsection{Timings}
The computation times are dominated by the time it takes to compute a solution to the binary integer program. We list details of this optimization for selected blocks in Tables~\ref{table:runtimes2} and \ref{table:runtimes1}.
Other components that contribute to the runtime are image processing to extract building-\facades (about 45 seconds per image), 
mesh processing to extract \sweepedges and \cleanprofiles (less than 20 seconds per block), grid-based regularization of \facade elements (less than 3 seconds per \facade), basic mass model construction (less than 10 seconds per block), and \facade element insertion into the mass models (less than 10 seconds per block).


 
\begin{table}[t!]
\begin{center}

\caption{\neu{Details for Figure~\ref{fig:detroitResults}, columns as in Table~\ref{table:runtimes2}.} }
\label{table:runtimes1}
\begin{tabular}{|c|c|c|c|c|c|} 

\hline
 Fig:row & location & $|\mathcal{C}|$ & $|\mathcal{S}|$ & vars & solve\\
                    & (lat,long) &     &  &          & time \\
 \hline\hline
 \ref{fig:detroitResults}:1 & 42.38458, & 9 & 5 & 196 & 0.01s \\ 
   &-82.95086  &   &   &     &      \\  
 \ref{fig:detroitResults}:2 & 42.38458, & 8 & 7 & 657 & 0.05s \\
 &-82.95084  &   &   &     &        \\   
 \ref{fig:detroitResults}:3 & 42.38587, & 6 & 4 & 165 & 0.00s \\
 &-82.95165  &   &   &     &        \\   
 \ref{fig:detroitResults}:4 & 42.38614 & 23 & 13 & 1,799 & 2.92s \\
 &-82.95125  &   &   &     &        \\   
 \ref{fig:detroitResults}:5 & 42.38350, & 37 & 14 & 1,494 & 0.3s \\ 
 &-82.94954  &   &   &     &        \\  
 \hline
\end{tabular}
\end{center}
\end{table}


\begin{figure}[b]
    \centering
  \def\svgwidth{\linewidth}  
    \input{../images/megafacades/problems.pdf_tex}
  \caption{{\it Limitations.} (Top)~Curved \facades can become over-fragmented during  
  sweep-line fitting and then adversely affect the \streetI\ analysis stage, resulting in missed \buildingfacade elements. 
  (Bottom)~Another limitation is handling buildings with curved profiles. }
  \label{fig:problems}
  \vnudge
\end{figure}

\subsection{Comparison}
We compared our work to other related algorithms in Figure~\ref{fig:comparis2}. As there exists no competing work to fuse multiple data sources, we limited our comparison to the processing of mass models. Therefore, we did not use {\GISd}s or \buildingfacades as input to any of the algorithms for this comparison; we used only the polygon soup meshes. To select competing work, we limited our choices to methods that had sourcecode available or where the authors helped us to generate results. The first method in our comparison is Poisson reconstruction~\cite{Kazhdan2006}, which can fill some smaller holes in the input, but the output looks similar to the input. Fitting a polygonal model using the Manhattan-world assumption~\cite{li2016manhattan} works well when the geometry conforms to such an assumption. However, we can see that over sloped roofs and within a larger block of buildings, the surface orientations vary too much, allowing the algorithm to produce good results on only one of the three inputs. Finally, we compare our method to structure-aware mesh decimation~\cite{salinas2015structure}, which also produces good results, but only a part of the model is simplified.


\begin{figure*}[t!]
    \centering
        \includegraphics[width=\linewidth]{../images/results/comparis2.png}
  \caption{{\it Comparison.} Columns left to right, input polygon soup mesh, Poisson reconstruction~\cite{Kazhdan2006}, Manhattan box fitting~\cite{li2016manhattan}, Structure-Aware Mesh Decimation~\cite{salinas2015structure} and our technique (without {\GISd}s or \buildingfacade inputs).}
  \label{fig:comparis2}
  \vnudge
\end{figure*}


\subsection{\LondonRS} 
Finally, we also provide results for a larger area in London consisting of 37 building blocks and 1,011 buildings (see Fig.~\ref{fig:results_london_big}). We used 738 images to find 2,716 \buildingfacades giving rise to 19,377 detected features.
We used a fixed computational budget of 1~hr for small blocks and 4~hrs for large blocks; the optimization returns the best solution found within the given time. A 40 core (10 $\times$ E5-2630) server was used for this example. 




\subsection{Limitations} 
Our system suffers from a few limitations. The PE representation of our mass models uses straight-line segments for \footprintpolygons and profiles, so we cannot correctly capture freeform buildings (e.g., buildings with a curved front or requiring a curved profile as in Figure~\ref{fig:problems}). \neu{In addition, our aggressive profile processing has the consequence that overhanging structures cannot be represented (e.g., bridges or balconies).} Another source of error is misclassifications of \facade imagery. This is particularly the case when our classifier encounters datasets with building styles for which it has not been trained.
%
We found datasets from certain European cities to be particularly challenging as the \streetI\ had to be obtained from narrow streets and alleys, resulting in strong perspective distortions. Other reasons for low accuracy classification results are very tall buildings, untrained features (e.g., fire escapes, buses, statues, etc.), or recessed floors that are not visible from street-level imagery.
%
While we expect that our classification results will continue to improve with access to more annotated training data, in the interim, allowing the user to correct mistakes would be a good alternative. \neu{Another observed failure case occurs when roof gutters do not align to detected \buildingfacade boundaries, as our optimization assumes such situations are noisy data.}
%
%
Finally, our core optimization relies on a BIP solver that globally combines the input data sets. This prevents us from developing an interactive system because the resulting optimization can run for multiple hours for larger city blocks. However, because the actual coupling is at the city-block level, the problem does not amplify with increasing city size as long as the complexity of the city blocks remains constant. 










\section{Conclusion}
\label{sec:conclusion}


We present a system to fuse partial and heterogeneous sources of data, specifically building footprints from GIS databases, polygonal meshes (polygon soup), and \streetI, to produce 
plausible \outputMs\ for densely-built \blocks.  Technically, we achieve this by formulating a binary integer program that {\em simultaneously} considers how to partition the \groundplane, assign profiles, and position \buildingfacades.
In the process, we globally balance information from  
incomplete and inconsistent input data to produce a semantically consistent \outputM. We evaluated our system on large scale datasets, spanning multiple urban blocks, to produce semantic results at a scale and quality not previously possible using state-of-the-art automated workflows. 
Incidentally, we introduced a new CNN for detecting \facade elements (e.g., windows, doors, etc.) on real-world images, and a mesh processing framework to decompose architectural meshes into footprints and profiles.

Our work opens up several future research directions. As an immediate next step, we would like to evaluate our CNN on other city datasets, and collect additional training data (i.e., labels) on \facade images from a wider range of cities to improve classification accuracy.  
Another interesting direction is to develop a semi-automatic system to allow users to edit inaccurate footprints, profiles, \buildingfacades, or \facade elements, to improve the output quality. For example, the user can mark a few smaller features, such as fire-escapes or air-conditioning units, which can then be used to refine city-specific feature detectors. In the longer-term, we envision a two-stage dynamic city-modeling tool, where a few city blocks are initially reconstructed using our proposed system. Once the models are approved by the user, the \outputM\ can be used to obtain a style description of buildings in the city. Such a description can then be used for wider-scale data integration, allowing us to handle large areas of missing data. Thus, the first round of results would act as a prior to synthesize missing information. This workflow would make it feasible to rapidly produce high-quality \outputMs\ of entire cities. 


\section*{Acknowledgements} 
\neu{
We would like to thank the many people who contributed to this paper; the reviewers, image labellers, and others who read manuscripts, each made valuable contributions. In particular, we thank Florent Lafarge, Pierre Alliez, Pascal M{\"u}ller, and Lama Affara for providing us with comparisons, software, and sourcecode, as well as Virginia Unkefer, Robin Roussel, Carlo Innamorati, and Aron Monszpart for their feedback. This work was supported by the ERC Starting Grant (SmartGeometry StG-2013-335373), KAUST-UCL grant (OSR-2015-CCF-2533), the KAUST Office of Sponsored Research (award No. OCRF-2014-CGR3-62140401), the Salt River Project Agricultural Improvement and Power District Cooperative Agreement No. 12061288, and the Visual Computing Center (VCC) at KAUST.}



\newpage


\if0

Pros:
\begin{itemize}
    \item Even if our results are not super accurate, the applications of creating ``plausible'' outputs are...
    \item The robustness of the PEs for parameterizing and reconstructing interesting mass models. The equivalent shape grammar would have been expensive to optimize. 
    \item we can represent interesting floorplans, such as courtyards or U-shaped buildings, each with arbitrary roofs.
    \item Procedural extrusions are unstable when adjacent edges in the plan are near-parallel and near-adjacent.
    \item Procedural extrusions with arbitrary plans and profiles are not guaranteed to terminate; we introduce measures to ensure they do.
    \item always architectural results. Mostly plausible results, such results are useful for illustration, parameter editing to fix discrepancies and large area reconstructions.
    \item PEs remove the problem of finding building roofs, but introduce the issues of working with non-rectangular facades.
    \item works with partial/missing data. 
    \item works with mixed and conflicting data sources (e.g. scaffolding in mesh but not in image, or vice-versa. Can use historical images to boost accuracy).
    \item the system is fully automatic
    \item Segnet = Magic.
\end{itemize}


Cons:
\begin{itemize}
    \item PEs only work well for building shaped things. Profile-distance function makes assumption that mesh has strong horizontal edges.
    \item It doesn't worked on curved walls.
    \item We assume correlation between roof-shapes and facades.
    \item data collection is burden, but sources such as Google Earth and Streetview provide an ample supply of mesh and photographic data. High quality GIS data for many regions can be found with OpenStreetMap.
    \item The optimization is slow. Future improvements in integer programming, may improve this.
    \item there is no guarantee that there will be a wall under any given window; in this case we remove the window. 
    \item The image feature detection only works on facade-types similar to those it has seen before.
\end{itemize}
\fi


\appendix 

\section{Representing Boolean Operators in a BIP}


In this appendix, we note that arbitrary Boolean relationships ($\land$, $\lor$, $\oplus$, $\neg$, $=$ etc.) can be encoded as IP constraints with additional variables and constraints (see \cite{optBook08}); such variables are omitted from the main text, but some examples are given in Table~\ref{table:ops}. %http://cs.stackexchange.com/questions/12102/express-boolean-logic-operations-in-zero-one-integer-linear-programming-ilp
%
%We leave the implementation of $\neg$ as future work. 
Modern IP solvers~\cite{gurobi} are very efficient at solving such trivially constrained sets of variables.
% 
Finally, we recall that the logical disjuction of a binary selection vector,
%
$$
r = \chi_1 \lor \dots \lor \chi_n,
$$
%
can be more efficiently implemented as a summation, given that only one element will take the value 1, as
%
$$
r = \sum_{i=1}^n \chi_i.
$$
 


\begin{table}[h]
\begin{center}
\caption{Expressing Boolean operations in a BIP.}
\label{table:ops}
\begin{tabular}{|c|c|c|c|} 
  \hline
  \small
expression & $c = a \land b$ &  $c = a \oplus b$ &  $c = a \lor b$ \\
\hline
 \hline
BIP encoding & $\begin{aligned} c &\ge a + b - 1\\ c &\le a\\ c &\le b, \end{aligned}$ &
$\begin{aligned}c &\le a + b\\ c &\ge a-b \\ c &\ge b-a\\ c &\le 2-a-b\end{aligned}$ &
$\begin{aligned}c &\le a + b\\c &\ge a\\c &\ge b\end{aligned}$ \\
 \hline
\end{tabular}
\end{center}
\end{table}




\if0
For example, we may calculate the value of $c = a \land b$ as
%
\neu{
\begin{align*}
c &\ge a + b - 1\\ 
c &\le a\\
c &\le b,
\end{align*}}
%
%
%
the value of $c = a \oplus b$ as
%
\begin{align*}
c &\le a + b\\
c &\ge a-b\\
c &\ge b-a\\
c &\le 2-a-b,
\end{align*}

and $c= a \lor b$ as
%
\begin{align*}
c &\le a + b\\
c &\ge a\\
c &\ge b.
\end{align*}
\fi
 
 
 
\section{Avoiding Bad Geometry}

\begin{wrapfigure}[8]{r}{0.4\columnwidth}
\vspace*{-.35in}
\hspace*{-.2in}
  \includegraphics[width=0.45\columnwidth]{../images/partition/bad.png}
\end{wrapfigure}
The ground tessellation, $\mathbb{G}$, is created by a variety of data sources. Hence, it can contain unlikely combinations of edge selections that we wish to avoid. For example, edges that are parallel, and in close proximity with one another, may create skinny \footprintpolygons, while pairs of edges with a small angle between them may produce pointed polygons. Such details are un-architectural, and we can optionally add a term to our optimization that penalizes undesirable pairs of edges within a polygon (this term was used in the \LondonRS~example shown in Figure~\ref{fig:teaser}).

We find pairs of edges within each face that we wish to penalize, $\te{bad}(\mathbb{G})$. This set contains pairs of edges that are approximately parallel, and less than 2.5m apart, or are adjacent with an angle less than $30^\circ$ (pairs of such lines are shown in pink and blue in the above inset). 
Entries from this set can be discouraged by only selecting one edge from each pair; we model such a penalty term as
%
$$O_7 (\{ \isEdge^k \}) := \sum_{(\edge_i,\edge_j) \in \te{bad}(\mathbb{G})} \isEdge^i \land  \isEdge^j $$
with a large weight of \neu{$\alpha_{7} =  0.5 \sum_{\edge_k \in \mathbb{G}}  \|\edge_k\|.$}

\pagebreak


%\section*{Acknowledgements}
%A company whose name rhymes with ``poodle'', for all the data.


%\bibliographystyle{acmsiggraph}
\bibliographystyle{ACM-Reference-Format}
%\nocite{*}
\bibliography{paper}
\end{document}
